\documentclass[12pt]{article}
\usepackage{amsmath}
\usepackage{booktabs}
\bibliographystyle{unsrt}
\usepackage{fancyhdr}
\pagestyle{fancy}
\usepackage[hidelinks]{hyperref}
\usepackage{lscape}

\setlength{\headheight}{15.2pt}
\pagestyle{fancy}
\renewcommand{\footrulewidth}{0.4pt}
\renewcommand{\headsep}{15pt}
\fancyfoot[LO,RE]{Alexandros Filotheou}

\title{HW3 \\ Alexandros Filotheou \\ DD2380 \\ Artificial Intelligence}
\date{}

\begin{document}
  \maketitle
  
Our time is the time of \textit{Civilization} as Oswald Spengler put it in his opus, \textit{The Decline of the West} \cite{TDotW}, which means that technical progress has surmounted the intellectual (or spiritual) culture. Frey and Osborne \cite{TFoE} deal with the fate of employment with regard to the deepening of technical progress, realised through the computerization of jobs. The effect is analogous to the type of ramifications the adoption of steam or electricity power had; human labour is substituted by technological advances and people are coerced to look for other types of jobs, which are still dependant on human labour. 

Computerization picks up after the effects that electrification had on the composition of employment; electrification exerted an evolutionary pressure on the employment environment, such that demand was for skilled blue and educated white collar workers \cite{TOoTc}, \cite{Allen}. People's jobs, divided into routine or non-routine intensive, and manual or cognitive, are susceptible to computerization in different degrees. However, advances in robots' manual dexterity, mobility and sensory, and in Machine Learning mean that more and more types of jobs are in danger of being hijacked by computers; not only manual and routine ones. In \cite{TFoE}, Frey and Osborne conclude that $47\%$ of their considered occupations of the U.S. labour market are likely to be automated by machines in the near future, while the fate of the rest depends on the breakthroughs on robotics and AI and, essentially, how much related they are to the human psyche.

With resistance to new technologies diminishing as the balance of power in society is skewed away from the people\footnote{From \cite{TFoE}: ``The balance between job conservation and technological progress therefore, to a large extent, reflects the balance of power in society, and how gains from technological progress are being distributed.".}, computerization, in my humble opinion at least, is going to march unobstructed\footnote{That does not necessarily mean that the rate of unemployment will follow the rate of births globally; developed nations that are computerization-intensive exhibit lower birth rates than the other nations, which still lack behind in adoption of new technologies \cite{DCFR}.}. However, that does not imply that unemployment will rise necessarily; the widespread adoption of electricity in employment in the twentieth century may have rendered some occupations deprecated, but gave rise to and spawned others that were not foreseeable at that time. For example, the transistor, the building block of modern computers, would only be possible only if the door to electrification was open. This facilitated the migration of labourers from one area of occupation to another and paved the way for AI. Just as electrification demanded skilled workers to work on complicated machinery as guides and overseers, maybe something analogous will happen with computerization. 

For sure though, given the findings of Frey and Osborne, people of low-wage and low-skill jobs will be soon replaced by machines, and they will have to migrate to jobs where creative and social skills are required. This might play the evolutionary pressure role needed to shift the landscape of occupation. However, the rise in supply might not be met by corresponding demand; this will undoubtedly render the occupation landscape barren for them.


The ethics of employment and its evolution are also subject to the balance of power within society. With respect to the advances of AI, ethics is an issue as long as he who has the upper hand makes it an issue. And as it seems, in many cases it is not made an issue. For example, the European Union has launched \textit{Horizon 2020}, a research framework which, among other things, ``[...] aims to keep older people active and independent for longer [...]" \cite{H20}. In a nutshell, home nursing for the elderly is effectively offshored to robots. Their tasks include health monitoring, exercising (for example mental exercises with regard to alzheimer patients) and providing such services to the elderly which up until now were handled by humans, experts or not. The quality of these services may increase and augment. For example not all carers can undertake all the tasks a robot can, while unlike them, it can provide care around the clock. Simultaneously, it will be cheaper to be cared for by a machine rather than a human. However, my ethical concerns lie exactly with the behaviour our younger generations towards their parents, the people who cared for them while they were in need, while they were unable to stand on their own two feet. First, this attitude is consequential and characterizes the overall egotistical trajectory the modern person has taken. Second, it is an indicator of the overall heartless and inhumane shape the psychology of the world is taking. Third, it does not take into account the psychology of the elderly; just their physical health. I suspect that human contact is more beneficial than that between a human and a machine. Fourth, it hurts and shows how much disrespect there seems to be towards us, as humans, irrespective of our age and towards our collective history. I have no doubt that there are exceptions; I am merely pointing out the trend and the composition of this attitude. For this particular example, elderly care, I see no opposition standing across the robotic mentality. Perhaps only the people that are the deepest involved, i.e. the elderly, may dismiss this as profoundly disrespectful and uncaring.


% Bibliography
\bibliography{bibliography}

\end{document}